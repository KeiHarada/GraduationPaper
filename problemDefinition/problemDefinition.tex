\documentclass[11pt,j4a]{jarticle}

\usepackage[dvips]{graphicx}

\usepackage{amsmath}
\usepackage[top=20truemm,bottom=30truemm,left=25truemm,right=25truemm]{geometry}
\title{進捗報告}
\author{原田 圭}
\begin{document}

\maketitle
\section{Problem Description}
以下に本論文で扱う問題の定義を行う.地理空間情報を持った$m$台のセンサの集合を
$\textit{\textbf{S}}=\{s_{1},s_{2},\ldots,s_{m}\}$とする.
各センサ$s_{i} \in \textit{\textbf{S}}(1 \leq i \leq m)$は位置$l_{i}$に設置されている.
また属性$a_{i}\in \textit{\textbf{A}},
\textit{\textbf{A}}=\{a_{1},a_{2},\ldots,a_{p}\}$を持つ.
$s_{i}$は時間領域$\textit{\textbf{T}}=\langle t_{1},t_{2},\ldots,t_{n} \rangle$
上の各$t_{j}$において$a_{i}$に対応する値を測定する.
ここで$t_{j}(1 \leq j \leq n)$はタイムスタンプを表し,これらは等間隔であるものとする.
タイムスタンプ$t_{j}$における$s_{i}$の測定値を$s_{i}[t_{j}]$と定義する.
さらに測定値の変化率を式(\ref{eq:changeRate})のように定義する.
\begin{equation}
\label{eq:changeRate}
r_{i}[t_{j}]=\frac{s_{i}[t_{j+1}]-s_{i}[t_{j}]}{t_{j+1}-t_{j}}
\end{equation}
\\\\
$\textbf{定義1}$ \hspace{2pt} ($\textbf{evolving}$) \hspace{2pt}
しきい値$\Theta = (\theta^{+},\theta^{-})$が与えられたとき
$r_{i}[t_{j}] \geq \theta^{+}$である場合$s_{i}$は$t_{j}$で正の$evolving$であるという.また
$r_{i}[t_{j}] \leq \theta^{-}$である場合$s_{i}$は$t_{j}$で負の$evolving$であるという.
\\\\
$\textbf{定義2}$ \hspace{2pt} ($\textbf{evolving区間}$) \hspace{2pt}
センサ$s_{i}$における$\textit{\textbf{T}}$内の連続的な部分区間
$\textit{\textbf{I}}=\langle t_{j},t_{j+1},\ldots,t_{j+l} \rangle$に対して,
$t_{j},t_{j+1},\ldots,t_{j+l}$が全て正の$evolving$,あるいは全て負の$evolving$である場合,
$\textit{\textbf{I}}$を$evolving$区間と呼ぶ.$t_{j},t_{j+1},\ldots,t_{j+l}$が全て正の$evolving$であるとき,
$\textit{\textbf{I}}$を正の$evolving$区間と呼ぶ.
また負の$evolving$であるとき,$\textit{\textbf{I}}$を負の$evolving$区間と呼ぶ.
\\\\
本研究の目的は地理的に近いセンサ集合に対して属性間の相関を発見することである.
そこでセンサ集合$\textit{\textbf{S}}$が与えられたとき,
地理的に近い関係にある$\textit{\textbf{S}}$の部分集合
および同じ属性をもつ$\textit{\textbf{S}}$の部分集合を以下のように定義する.
\\\\
$\textbf{定義3}$ \hspace{2pt} ($\textbf{近傍集合}$) \hspace{2pt}
しきい値$h$と$\textit{\textbf{S}}$の部分集合$\textit{\textbf{G}} \subseteq \textit{\textbf{S}}$
が与えられたとき,
$\forall s \in \textit{\textbf{G}}, \exists s' \in \textit{\textbf{G}}-\{s\}
\hspace{2pt} s.t. \hspace{2pt} dist(s,s') \leq h$であるならば$\textit{\textbf{G}}$
は地理的に近い関係にあるセンサ集合であると定義する.このとき$\textit{\textbf{G}}$を近傍集合と呼ぶ.
ここで$dist(s,s')$は$s$と$s'$の空間的な距離を表す.
\\\\
$\textbf{定義4}$ \hspace{2pt} ($\textbf{同属性集合}$) \hspace{2pt}
センサの属性$a$と$\textit{\textbf{S}}$の部分集合$\textit{\textbf{S}}_{a} \subseteq \textit{\textbf{S}}$
が与えられたとき,$\forall s \in \textit{\textbf{S}}_{a}$が同じ属性$a$をもつならば,$\textit{\textbf{S}}_{a}$
を$a$の同属性集合と呼ぶ.
\\\\
$\textbf{定義5}$ \hspace{2pt} ($\textbf{近傍同属性集合}$) \hspace{2pt}
属性$a$の同属性集合$\textit{\textbf{S}}_{a}$の部分集合$\textit{\textbf{G}}_{a} \in \textit{\textbf{S}}_{a}$が与えられたとき,
$\textit{\textbf{G}}_{a}$が近傍集合であるならば,
$\textit{\textbf{G}}_{a}$を近傍同属性集合と呼ぶ.
\newpage
$\textbf{定義6}$ \hspace{2pt} ($\textbf{空間的な共起}$) \hspace{2pt}
近傍集合を$\textit{\textbf{G}}$,$\textit{\textbf{G}}$内のセンサ$s_{i} \in \textit{\textbf{G}}$が
$t_{j} \in \textit{\textbf{T}}$で$evolving$であるためのしきい値を$\Theta$とする.
あるタイムスタンプ$t_{j}$に対して,
$\forall s_{i} \in \textit{\textbf{G}},r_{i}[t_{j}] \geq \theta^{+}$
であるとき$t_{j}$は$\Theta$に正で共起するといい,$t_{j} \xrightarrow{+} \Theta$と表す.また
$\forall s_{i} \in \textit{\textbf{G}}, r_{i}[t_{j}] \leq \theta^{-}$
であるとき$t_{j}$は$\Theta$に負で共起するといい,$t_{j} \xrightarrow{-} \Theta$と表す.
正または負で共起するタイムスタンプの集合を
$E(\textit{\textbf{G}})=\{t_{j} \in \textit{\textbf{T}}|t_{j} \xrightarrow{+} \Theta \lor t_{j} \xrightarrow{-} \Theta\}$と表す.
\\\\
$\textbf{定義7}$ \hspace{2pt} ($\textbf{属性間の空間的な共起}$) \hspace{2pt}
二つの近傍同属性集合$\textit{\textbf{G}}_{a_{1}},\textit{\textbf{G}}_{a_{2}}(a_{1} \neq a_{2})$について
$s_{i} \in \textit{\textbf{G}}_{a_{1}}$が$t_{j} \in \textit{\textbf{T}}$で
$evolving$であるためのしきい値を$\Theta_{a_{1}}$,
$s_{i} \in \textit{\textbf{G}}_{a_{2}}$が$t_{j} \in \textit{\textbf{T}}$で
$evolving$であるためのしきい値を$\Theta_{a_{2}}$とする.
$\textit{\textbf{G}}_{a_{1}} \cup \textit{\textbf{G}}_{a_{2}}$が近傍集合であるとき,
$P_{a_{1},a_{2}}=\{t_{j} \in E(\textit{\textbf{G}}_{a_{1}}) \cap E(\textit{\textbf{G}}_{a_{2}})|
(t_{j} \xrightarrow{+} \Theta_{a_{1}} \land t_{j} \xrightarrow{+} \Theta_{a_{2}}) \lor
(t_{j} \xrightarrow{-} \Theta_{a_{1}} \land t_{j} \xrightarrow{-} \Theta_{a_{2}})\}$
を$\textit{\textbf{G}}_{a_{1}},\textit{\textbf{G}}_{a_{2}}$間の正の空間的な共起と呼ぶ.
また$N_{a_{1},a_{2}}=\{t_{j} \in E(\textit{\textbf{G}}_{a_{1}}) \cap E(\textit{\textbf{G}}_{a_{2}})|
(t_{j} \xrightarrow{+} \Theta_{a_{1}} \land t_{j} \xrightarrow{-} \Theta_{a_{2}}) \lor
(t_{j} \xrightarrow{-} \Theta_{a_{1}} \land t_{j} \xrightarrow{+} \Theta_{a_{2}})\}$
を$\textit{\textbf{G}}_{a_{1}},\textit{\textbf{G}}_{a_{2}}$間の負の空間的な共起と呼ぶ.
\\\\
$\textbf{定義8}$ \hspace{2pt} ($\textbf{属性間の相関}$) \hspace{2pt}
二つの近傍同属性集合$\textit{\textbf{G}}_{a_{1}},\textit{\textbf{G}}_{a_{2}}(a_{1} \neq a_{2})$間の正の空間的な共起を$P_{a_{1},a_{2}}$,負の空間的な共起を$N_{a_{1},a_{2}}$とする.
しきい値$\psi$が与えられたとき,$|P_{a_{1},a_{2}}| \geq \psi \hspace{2pt}$であるならば,
属性$a_{1},a_{2}$間にセンサ群$\textit{\textbf{G}}_{a_{1}},\textit{\textbf{G}}_{a_{2}}$上で正の相関があるという.
また,$|N_{a_{1},a_{2}}| \geq \psi \hspace{2pt}$であるならば,
属性$a_{1},a_{2}$間にセンサ群$\textit{\textbf{G}}_{a_{1}},\textit{\textbf{G}}_{a_{2}}$上で負の相関があるという.

\end{document}
